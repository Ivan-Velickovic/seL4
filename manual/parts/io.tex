%
% Copyright 2014, General Dynamics C4 Systems
%
% SPDX-License-Identifier: GPL-2.0-only
%

\chapter{\label{ch:io}Hardware I/O}

\section{Interrupt Delivery}
\label{sec:interrupts}

Interrupts are delivered as notifications. A thread
may configure the kernel to signal a particular \obj{Notification}
object each time a certain interrupt triggers. Threads may then wait for
interrupts to occur by calling \apifunc{seL4\_Wait}{sel4_wait} or
\apifunc{seL4\_Poll}{sel4_poll} on
that \obj{Notification}.


\obj{IRQHandler} capabilities represent the ability of a thread to
configure a certain interrupt. They have three methods:

\begin{description}
    \item[\apifunc{seL4\_IRQHandler\_SetNotification}{irq_handlersetnotification}]
    specifies the \obj{Notification} the kernel should
    \apifunc{signal}{sel4_signal} when an interrupt occurs. A driver
    may then call \apifunc{seL4\_Wait}{sel4_wait} or \apifunc{seL4\_Poll}{sel4_poll}
    on this notification to
    wait for interrupts to arrive.

    \item[\apifunc{seL4\_IRQHandler\_Ack}{irq_handleracknowledge}]
    informs the kernel that the userspace driver has finished processing
    the interrupt and the microkernel can send further pending or new
    interrupts to the application.

    \item[\apifunc{seL4\_IRQHandler\_Clear}{irq_handlerclear}]
    de-registers the \obj{Notification} from the \obj{IRQHandler} object.
\end{description}

When the system first starts, no \obj{IRQHandler} capabilities are
present. Instead, the initial thread's CSpace contains a single
\obj{IRQControl} capability. This capability may be used to produce
a single \obj{IRQHandler} capability for each interrupt available in the
system. Typically, the initial thread of a system will determine which
IRQs are required by other components in the system, produce an
\obj{IRQHandler} capability for each interrupt, and then delegate the
resulting capabilities as appropriate. Methods on \obj{IRQControl} can
be used for creating \obj{IRQHandler} capabilities for interrupt sources.

\ifxeightsix
\section{x86-Specific I/O}

\subsection{Interrupts}
\label{sec:x86_interrupts}

In addition to managing \obj{IRQHandler} capabilities, x86 platforms require
the delivery location in the CPU vectors to be configured. Regardless of where
an interrupt comes from (IOAPIC, MSI, etc) it must be assigned a unique vector
for delivery, ranging from VECTOR\_MIN to VECTOR\_MAX. The rights to allocate
a vector are effectively given through the \obj{IRQControl} capability and can
be considered as the kernel outsourcing the allocation of this namespace to
user level.

\begin{description}
    \item[\apifunc{seL4\_IRQControl\_GetIOAPIC}{x86_irq_control_get_io_apic}] creates
    an \obj{IRQHandler} capability for an IOAPIC interrupt

    \item[\apifunc{seL4\_IRQControl\_GetMSI}{x86_irq_control_get_msi}] creates
    an \obj{IRQHandler} capability for an MSI interrupt
\end{description}

\subsection{I/O Ports}
\label{sec:ioports}

On x86 platforms, seL4 provides access to I/O ports to user-level threads.
Access to I/O ports is controlled by \obj{IO Port} capabilities. Each
\obj{IO Port} capability identifies a range of ports that can be accessed with
it. Reading from I/O ports is accomplished with the
\apifunc{seL4\_X86\_IOPort\_In8}{x86_io_port_in8},
\apifunc{seL4\_X86\_IOPort\_In16}{x86_io_port_in16}, and
\apifunc{seL4\_X86\_IOPort\_In32}{x86_io_port_in32} methods, which
allow for reading of 8-, 16- and 32-bit quantities.
Similarly, writing to I/O ports is accomplished with the
\apifunc{seL4\_X86\_IOPort\_Out8}{x86_io_port_out8},
\apifunc{seL4\_X86\_IOPort\_Out16}{x86_io_port_out16}, and
\apifunc{seL4\_X86\_IOPort\_Out32}{x86_io_port_out32} methods.
Each of these methods takes as arguments an \obj{IO Port} capability
and an unsigned integer~\texttt{port}, which indicates the I/O port to read from
or write to, respectively.
In each case, \texttt{port} must be within the range of I/O ports identified
by the given \obj{IO Port} capability in order for the method to succeed.

The I/O port methods return error codes upon failure.
A \texttt{seL4\_IllegalOperation} code is returned if port access is
attempted outside the range allowed by the \obj{IO Port} capability.
Since invocations that
read from I/O ports are required to return two values -- the value read
and the error code -- a structure containing two members, \texttt{result}
and \texttt{error}, is returned from these API calls.

At system initialisation, the initial thread's \obj{CSpace} contains the
\obj{IOPortControl} capability, which can be used to \apifunc{seL4\_X86\_IOPort\_Issue}{x86_ioport_issue}
\obj{IO Port} capabilities to sub ranges of I/O ports. Any range that is issued
may not have overlap with any existing issued \obj{IO Port} capability.

\subsection{I/O Space}
\label{sec:iospace}

I/O devices capable of DMA present a security risk because the CPU's MMU
is bypassed when the device accesses memory. In seL4, device drivers run
in user space to keep them out of the trusted computing base.
A malicious or buggy device driver may, however, program the device to
access or corrupt memory that is not part of its address space, thus
subverting security. To mitigate this threat, seL4 provides support for
the IOMMU on Intel x86-based platforms. An IOMMU allows memory to be
remapped from the device's point of view. It acts as an MMU for the
device, restricting the regions of system memory that it can access.
More information can be obtained from Intel's IOMMU documentation \cite{extra:vtd}.

Two new objects are provided by the kernel to abstract the IOMMU:
\begin{description}

    \item[\obj{IOSpace}] This object represents the address space associated
    with a hardware device on the PCI bus. It represents the right to
    modify a device's memory mappings.

    \item[\obj{IOPageTable}] This object represents a node in the multilevel
    page-table structure used by IOMMU hardware to translate hardware
    memory accesses.

\end{description}

\obj{Page} capabilities are used to represent the actual frames that are
mapped into the I/O address space. A \obj{Page} can be mapped into
either a \obj{VSpace} or an \obj{IOSpace} but never into both at the same time.

\obj{IOSpace} and \obj{VSpace} fault handling differ significantly.
\obj{VSpace} page faults are redirected to the thread's exception handler (see \autoref{sec:faults}),
which can take the
appropriate action and restart the thread at the faulting instruction.
There is no concept of an exception handler for an \obj{IOSpace}. Instead, faulting
transactions are simply
aborted; the device driver must correct the cause of the fault and retry
the DMA transaction.

An initial master \obj{IOSpace} capability is provided in the initial thread's
CSpace. An \obj{IOSpace} capability for a specific device is created by
using the \apifunc{seL4\_CNode\_Mint}{cnode_mint} method, passing the
PCI identifier of the device as the low 16 bits of the \texttt{badge} argument, and
a Domain ID as the high 16 bits of the \texttt{badge} argument.
PCI identifiers are explained fully in the PCI specification
\cite{Shanley:PCISA}, but are briefly described here. A PCI identifier is
a 16-bit quantity.  The first 8 bits identify the bus that the device is on.
The next 5 bits are the device identifier: the number of the device on
the bus. The last 3 bits are the function number. A single device may
consist of several independent functions, each of which may be addressed
by the PCI identifier.
Domain IDs are explained fully in the Intel IOMMU documentation \cite{extra:vtd}.
There is presently no way to query seL4 for how many Domain IDs are supported by
the IOMMU and the \apifunc{seL4\_CNode\_Mint}{cnode_mint} method will fail if an
unsupported value is chosen.

The IOMMU page-table structure has three levels.
Page tables are mapped into an \obj{IOSpace} using the \apifunc{seL4\_X86\_IOPageTable\_Map}{x86_io_page_table_map} method.
This method takes the \obj{IOPageTable} to map, the \obj{IOSpace} to map into
and the address to map at. Three levels of page tables must be mapped before
a frame can be mapped successfully. A frame is mapped with the
\apifunc{seL4\_X86\_Page\_MapIO}{x86_page_map_io} method whose parameters are analogous to
the corresponding method that maps \obj{Page}s into \obj{VSpaces} (see \autoref{ch:vspace}),
namely \apifunc{seL4\_X86\_Page\_Map}{x86_page_map}.

Unmapping is accomplished with the usual unmap (see \autoref{ch:vspace}) API
call,
\apifunc{seL4\_X86\_Page\_Unmap}{x86_page_unmap}.

More information about seL4's IOMMU abstractions can be found in \cite{Palande:M}.
\fi

\section{Arm-Specific I/O}

\subsection{Arm SMMU version 2.0}
\label{sec:smmuv2}


seL4 provides an API for programming the Arm System MMU (SMMU) version 2.0,
which allows system software to manage access rights and address translation for
devices that can initiate direct memory accesses (DMA).

An Arm SMMU v2.0 implementation allows device memory transactions to be associated
with an identifier (StreamID) that is used to direct the transaction through a
SMMU translation context bank (CB). A translation context bank can perform
address translation, memory protection and memory attribute transformation.
The standard specifies different types of address translations that correspond
to stages in the ArmV8 virtual memory system architecture such as either
non-secure  EL0, EL1 first and second stage translations, Hyp mode translations
or secure mode translations. It is possible to associate different StreamIDs
with the same context bank and it is possible to share address translation
tables between a context bank and software MMU address space if the stage and type
of translation is the same.

Faults that occur when a memory transaction conflicts with a StreamID or CB
configuration happen asynchronously with respect to a processor element's
execution. When this occurs an interrupt is used to allow a PE to handle the
SMMU fault. Faults are reported through registers in the SMMU that can be
queried in an interrupt handler.

TLB maintenance operations are required to keep SMMU translation caches
consistent when there are changes to any valid page table mapping entries.

An SMMU implementation usually has a maximum number of StreamIDs that it supports.
The specificiation allows StreamIDs to be up to 16bits wide. There are also a
fixed number of context banks, up to a maximum of 128. Context banks can
be generic or support only a single address translation stage. This information
is reported by ID registers in each implementation.

The seL4 API allows system software to manage an SMMU by assigning StreamIDs to
context banks, bind context banks to page translation structures, implement
SMMU fault handling and also perform explicit TLB maintenance.
This allows system software to ensure that a device is only able to access and
modify memory contents that it has been explicitly given access to and allow
devices to be presented with a virtualised address space for performing DMA.

All the StreamIDs and context banks are accessible via capabilities. Control
capabilities are used to create capabilities referring to each StreamID and
context bank in a system. The kernel tracks the allocation of StreamIDs and
context banks with two static CNodes, one for each resource type. These CNodes
track which VSpace a context bank has bound to it, and which context bank a
StreamID is bound to.

The capabilities allow access control policies to be implemented by a user thread.
When StreamID, context bank capabilities are revoked, the kernel will disable
the context banks or StreamID mappings.

TLB maintenance is handled by the kernel via tracking which context banks are
associated with a particular VSpace. Any TLB maintenance operations that the
kernel performs on VSpace invocations are also applied to associated context
banks.

SMMU fault handling is delegated to user level via invocations that allow fault
statuses to be queried and cleared for each context bank and for the SMMU globally.
SMMU fault interrupts can be handled the same as other platform level interrupts.

The kernel implementation only uses translation stages matching what translation
the kernel is performing for VSpace objects. When seL4 is operating in EL1,
the SMMU only uses stage 1 translation (ASID), that is "stage 1 with stage 2
bypass" in the context bank attribute configuration. When hypervisor mode is
enabled, and seL4 is operating in EL2, the SMMU only does stage 2 translations.

Four capabilities types provide access to SMMU resources:
\begin{description}
    \item[\obj{seL4\_ARM\_SID}] A capability granting access to a single
        transaction stream, which can be used to bind and unbind a stream to a
        single context bank.
    \item[\obj{seL4\_ARM\_CB}] A capbility representing a single specific context
        bank. It can be used to bind and unbind a VSpace to assign what page
        tables the context bank should use for translation, assign StreamIDs and
        process context bank faults.
    \item[\obj{seL4\_ARM\_SIDControl}] A control capability which can be used to
        create \obj{seL4\_ARM\_SID} capabilities to specific transaction streams.
        The \obj{seL4\_ARM\_SIDControl} cap is used for managing rights on
        StreamID configurations. This capability is provided in the initial
        thread's CSpace.
    \item[\obj{seL4\_ARM\_CBControl}] A control capability that can be used to
        derive \obj{seL4\_ARM\_CB} capabilities. The \obj{seL4\_ARM\_CBControl}
        cap is used for managing rights on context bank configurations.
        This capability is provided in the initial thread's CSpace.
\end{description}


\subsubsection{Creating \obj{seL4\_ARM\_SID} capabilities}
\label{sec:smmuv2-creating-sel4-arm-sid-capabilities}

The Arm SMMU 2.0 specification doesn't specify how StreamIDs need to correspond
to different devices. Each platform can define its own policy for how StreamIDs
are allocated. A \obj{seL4\_ARM\_SIDControl} capability can be used to create
a capability to any valid StreamID for the SMMU and delegate access to other
tasks in the system.

\begin{description}
\item[\apifunc{seL4\_ARM\_SIDControl\_GetSID}{arm_sid_controlgetsid}] uses the
    \obj{seL4\_ARM\_SIDControl} capability to create a new \obj{seL4\_ARM\_SID}
    capability that represents a single StreamID.  This new capbility is placed
    in the provided slot.  It is expected that whatever thread controls an
    \obj{seL4\_ARM\_SIDControl} capability knows about how StreamIDs are
    allocated in a system.
\end{description}

The Arm SMMU 2.0 specification descibes many ways of associating StreamIDs with
context banks. Currently only direct mapping of a StreamID to a context bank is
supported.

\subsubsection{Creating \obj{seL4\_ARM\_CB} capabilities}
\label{sec:smmuv2-creating-sel4-arm-cb-capabilities}

Each context bank allows the SMMU to maintain an active translation context with
it's own registers for holding context specific information. An SMMU has a fixed
number of context banks available for use and these are allocated using the
\obj{seL4\_ARM\_CBControl} capability.

\begin{description}

\item[\apifunc{seL4\_ARM\_CBControl\_GetCB}{arm_cb_controlgetcb}] uses the
    \obj{seL4\_ARM\_CBControl} capability to create a new \obj{seL4\_ARM\_CB}
    capability that represents a single context bank.  This new capability is
    placed in the provided slot.  It is expected that whatever thread controls a
    \obj{seL4\_ARM\_CBControl} capability has knowledge of the properties of each
    context bank that each index refers to.
\end{description}


\subsubsection{Configuring context banks}
\label{sec:smmuv2-configuring-context-banks}

By providing a \obj{seL4\_ARM\_CB} cap, a user-level thread can configure the
VSpace used by the bank with the following API:

\begin{description}
    \item[\apifunc{seL4\_ARM\_CB\_AssignVspace}{arm_cb_assignvspace}] configures
        the context bank to use the provided VSpace root for translations.
    \item[\apifunc{seL4\_ARM\_CB\_UnassignVspace}{arm_cb_unassignvspace}] removes
        the configured VSpace and invalidates the TLB.
\end{description}

The SMMU-v2 uses the same paging structure as the MMU (AArch\_64 and AArch\_32
formats). Therefore, there is no need to provide a new set of page structure caps
nor a separate set of map and unmap functions. To manage the assignment, the
kernel has an internal CNode, called smmuStateCBNode, that stores copies of the
\obj{VSpace\_cap} created by executing the above API. The copy of the
\obj{VSpace\_cap} contains its assigned ContextBank number. Therefore the kernel
can conduct context bank invalidation if the \obj{VSpace\_cap} is revoked.


\subsubsection{Configuring streams (transactions)}
\label{sec:smmuv2-configuring-streams-transactions}

A user-level thread can bind a context bank with an \obj{seL4\_ARM\_SID}
capability with:
\begin{description}
    \item[\apifunc{seL4\_ARM\_SID\_BindCB}{arm_sid_bindcb}] configures the stream
        to use given context bank for translation. To simplify the process, the
        binding also enables the stream ID.  \obj{seL4\_ARM\_SID\_BindCB}
        generates a copy of the \obj{seL4\_ARM\_CB} cap in kernel's internal
        CNode. This allows the stream ID to be disabled if the
        \obj{seL4\_ARM\_CB} cap is revoked.
    \item[\apifunc{seL4\_ARM\_SID\_UnbindCB}{arm_sid_unbindcb}] removes the
        \obj{seL4\_ARM\_CB} cap from the kernel's internal CNode and disables
        the stream ID. The kernel provides this API for the conveniences of
        sharing a stream ID among multiple VSpaces.
\end{description}

If there are any exceptions after the stream ID is enabled, the user-level
software should use the fault handling mechanisms to resolve them.



\subsubsection{Copying and Deleting caps}
\label{sec:smmuv2-copying-and-deleting-caps}
The kernel allows copying both \obj{ARM\_SID} cap and \obj{seL4\_ARM\_CB} cap.
This allows capabilites to be delegated to different threads.
The kernel does not allow copying neither the \obj{seL4\_ARM\_SIDControl} nor
the \obj{seL4\_ARM\_CBControl} capabilities.

Deleting a \obj{seL4\_ARM\_CB} cap that contains a valid capBindSID field will:
\begin{itemize}
    \item invalidate the streamID to ContextBank assignment in hardware.
\end{itemize}

Deleting the last \obj{seL4\_ARM\_CB} cap will:
\begin{itemize}
    \item perform an \apifunc{seL4\_ARM\_CB\_UnassignVspace}{arm_cb_unassignvspace},
    removing any configured VSpace,
    \item conduct a TLB invalidation.
\end{itemize}

Similarly, deleting a VSpace\_cap that contains assigned context bank number will:
\begin{itemize}
    \item invalidate the context bank
    \item conduct a TLB invalidation
\end{itemize}

Deleting the last ARM\_SID cap will:
\begin{itemize}
    \item Perform an \apifunc{seL4\_ARM\_SID\_UnbindCB}{arm_sid_unbindcb},
        (deleting the copy of the assigned \obj{seL4\_ARM\_CB} cap)
    \item Disable the stream ID.
\end{itemize}

\subsubsection{TLB invalidation}
\label{sec:smmuv2-tlb-invalidation}
The kernel is expected to perform all required SMMU TLB maintenance operations
as part of the API implementation.  In addition, the kernel provides two system
calls for explicitly performing invalidations:
\begin{description}
    \item[\apifunc{seL4\_ARM\_CBControl\_TLBInvalidateAll}{arm_cb_controltlbinvalidate}]
        invalidates all TLB entries in all
    context banks.
    \item[\apifunc{seL4\_ARM\_CB\_TLBInvalidate}{arm_cb_tlbinvalidate}]
        invalidates all TLB entries in a context bank.
\end{description}

The kernel does not impose any restrictions on how a VSpace is used by user-level
applications, hence a VSpace can be shared by normal threads and drivers. Sharing
a VSpace between threads and drivers also means sharing all mappings in that
VSpace between MMUs in CPU cores and SMMU used by device transactions. Moreover,
multiple context banks in SMMU can share a VSpace. Therefore, maintaining the
coherency between the TLB in MMU and the TLB in SMMU's context banks is important.

The kernel keeps a record of Vspace's usage in context banks in SMMU by
maintaining: the number of context banks using a given ASID, and the ASID that a
given context bank is using. There are a few reasons behind this design.
\begin{itemize}
\item First, the ASID is efficient for representing a VSpace. In seL4, each VSpace has
an ASID which is assigned before the VSpace is ready to be used and will never
change until the VSpace is deleted. Recording how many context banks are using a
VSpace's ASID is equivalent to recording the VSpace's usage in context banks.
\item Second, all TLB invalidation operation requires knowledge of the ASID. There are
two types of TLB invalidation operations: invalidating a page table entry using
its ASID (triggered by updating a page table entry, e.g. unmapping a page), and
invalidating all mappings of an ASID (triggered by deleting a VSpace).
\item Third, the kernel can easily find a context banks' ASID on all occasions, which is
useful to either conduct TLB invalidation requests or unassign VSpace from a
context bank.
\end{itemize}

By knowing how many context banks are using an ASID, the kernel can easily check
in every TLB invalidation operation and invoke TLB invalidation in SMMU if the
value is not zero. In SMMU's TLB invalidation operation, the kernel searches the
context banks using the ASID, and conducts TLB invalidation in those context banks.

Ideally, the SMMU shares the same ASID or VMID name space with the rest of the
system. This allows the SMMU to maintain TLB coherency by listening for TLB
broadcasting messages. This means the context banks should be configured with
the correct ASID or VMID when the StreamID is enabled. This is not a problem for
stage 1 translation, as there are a large number of ASID bits and an ASID can be
assigned to a vspace root with existing APIs. However, the VMID used in stage 2
only has 8 bits, and the kernel allocates them on demand and can reclaim a
VSpace's hardware ASID to reuse if there are more VSpaces than available ASIDS.
While it is possible to do this when the VSpace is only used in an MMU, it is
not possible with multiple active context banks.
Due to this, the context bank in SMMU cannot be configured with the correct VMID.
Currently, the SMMU driver uses private VMID space, and uses the context bank
number as the corresponding VMID number.



\subsubsection{Fault handling}
\label{sec:smmuv2-fault-handling}
The number of IRQs used for reporting transaction faults is hardware dependent.
There are two kinds of faults: global faults ( general configuration and
transaction faults), or context bank faults. For transaction faults, the SMMU
reports faulty stream IDs. The global faults reports:
\begin{itemize}
    \item Invalid context fault.
    \item Unidentified stream fault.
    \item Stream match conflict fault.
    \item Unimplemented context bank fault.
    \item Unimplemented context interrupt fault.
    \item Configuration access fault.
    \item External fault.
\end{itemize}
Each context bank contains registers to report faults on address translation, for
example, faulty addresses, or permission errors. The SMMU driver identifies the
cause of a fault by first reading the global fault registers (one state register
and three fault syndrome registers), then by reading corresponding context bank
fault registers. Note, the SMMU reports the faulty transaction (stream) ID,
which can be used to identify its context bank ID.

\begin{itemize}
\item System assumption: Both the SMMU's IRQ handler and the owner of the
    \obj{seL4\_ARM\_SIDControl} cap (controlling stream ID distributions) are trusted.
\item SMMU interrupts are handled as same as other IRQs, i,e, the kernel does not
    treat the SMMU IRQs special, reporting the interrupt via IRQ notifications.
\item The kernel provides a API for reading the global fault registers:
    \apifunc{seL4\_ARM\_SIDControl\_GetFault}{arm_sid_controlgetfault}. Because
    the IRQ notification can only deliver information via the badge, the owner
    of the StreamControl\_cap can retrieve more information via this API.
\item If the fault is related to a transaction, the owner of the
    \obj{seL4\_ARM\_SIDControl} cap will notify the holder of the corresponding
    stream ID cap, which should also have a copy of the context bank cap bound to
    this transaction.
\item The kernel provides an API for reading the context bank fault registers:
    \apifunc{seL4\_ARM\_CB\_CBGetFault}{arm_cb_getfault}, used by a context bank
    cap holder (the \obj{seL4\_ARM\_CB} cap holder).
\item Once the fault handling is done, the server can call
    \apifunc{seL4\_ARM\_CB\_CBClearFault}{arm_cb_clearfault} to clear the fault
    status on a context bank, and
    \apifunc{seL4\_ARM\_SIDControl\_ClearFault}{arm_sid_controlclearfault}
    to clear the fault status on SMMU.
\end{itemize}

