%
% Copyright 2014, General Dynamics C4 Systems
%
% SPDX-License-Identifier: GPL-2.0-only
%

\chapter{\label{ch:vspace}Address Spaces and Virtual Memory}

A virtual address space in seL4 is called a VSpace. Similarly to a CSpace (see \autoref{ch:cspace}),
a VSpace is composed of objects provided by the kernel. Unlike CSpaces, objects for managing virtual
memory correspond to those of the hardware and each architecture defines its own object types for
paging structures. Also unlike CSpaces, we call only the top-level paging structure a VSpace object.
It provides the top-level authority to the VSpace.

Common to all architectures is the \obj{Frame}, representing a frame of physical memory. Frame
objects are manipulated via \obj{Page} capabilities, which represents both authority to the frame,
as well as to the virtual memory mapping, i.e.\ the page, when mapped. The kernel also provides
\obj{ASID Pool} objects and \obj{ASID Control} invocations for tracking the status of address space
identifiers for VSpaces.

These VSpace-related objects are sufficient to implement the hardware data structures required to
create, manipulate, and destroy virtual memory address spaces. As usual, the manipulator of a
virtual memory space needs the appropriate capabilities to the required objects.

\section{Objects}

\subsection{Hardware Virtual Memory Objects}

Each architecture has a top-level paging structure (level 0) and a number of intermediate levels.
When referring to it generically, we call this top-level paging structure the VSpace object. The
seL4 object type that implements the VSpace object is architecture dependent. For instance on
AArch32, a VSpace is represented by the PageDirectory object and on x64 by a PML4 object.

In general, each paging structure at each level contains slots where either the next level paging
structure or a frame of memory can be mapped. The level of the paging structure determines the size
of the frame. The size and type of structure at each level, and the number of bits in the virtual
address resolved for that level are hardware defined.

The seL4 kernel provides methods for operating on these hardware paging structures including mapping
and cache operations. Mapping operations are invoked on the capability to the object being mapped.
For example, to map a level 2 paging structure at a specific virtual address, we can invoke the map
operation on the capability to the level 2 object and provide the virtual address as well as the
capability to the level 1 object as arguments.

If the previous level (level 1 in the example) is not itself already mapped, the mapping operation
will fail. Developers need to create and map all paging structures, the kernel does not
automatically create intermediate levels.

In general, the VSpace object (the top-level paging structure) has no invocations for mapping, but
is used as an argument to several other virtual-memory related object invocations. For some
architectures, the VSpace object provides cache operation invocations. This allows simpler
policy options: a process that has delegated a VSpace capability (e.g.\ to a page directory on
AArch32) can conduct cache operations on all frames mapped from that capability without needing
access to those capabilities directly.

The rest of this section details the paging structures for each architecture.

\subsubsection{IA-32}

On IA-32, the VSpace object is implemented by the \obj{PageDirectory} object, which covers the
entire 4\,GiB range in the 32-bit address space, and forms the top-level paging structure. Second
level page-tables (\obj{PageTable} objects) each cover a 4\,MiB range. Structures at both levels
are indexed by 10 bits in the virtual address.

\begin{tabularx}{\textwidth}{Xlll} \toprule
\emph{Object}          & \emph{Address Bits} & \emph{Level} & \emph{Methods} \\ \midrule
\texttt{PageDirectory} & 22---31             & 0            & \autoref{group__x86__seL4__X86__PageDirectory} \\
\texttt{PageTable}     & 12---21             & 1            & \autoref{group__x86__seL4__X86__PageTable} \\
\bottomrule
\end{tabularx}

\subsubsection{x64}

On x86-64, the VSpace object is implemented by the \obj{PML4} object. Three further levels of
paging structure are defined, as shown in the table below. All structures are indexed by 9 bits of
the virtual address.

\begin{tabularx}{\textwidth}{Xlll} \toprule
\emph{Object}          & \emph{Address Bits} & \emph{Level} & \emph{Methods} \\ \midrule
\texttt{PML4}          & 39---47             & 0            & None \\
\texttt{PDPT}          & 30---38             & 1            & \autoref{group__x86__64__seL4__X86__PDPT} \\
\texttt{PageDirectory} & 21---29             & 2            & \autoref{group__x86__seL4__X86__PageDirectory} \\
\texttt{PageTable}     & 12---20             & 3            & \autoref{group__x86__seL4__X86__PageTable} \\
\bottomrule
\end{tabularx}

\subsubsection{AArch32}

Like IA-32, Arm AArch32 implements the VSpace object with a \obj{PageDirectory} object which
covers the entire 4\,GiB address range.  The second-level structures on AArch32 are
\obj{PageTable} objects. The address range they cover is configuration-dependent: 1\,MiB
(20 address bits) for standard configurations, and 2\,MiB (21 address bits) for hypervisor
configurations.

\begin{tabularx}{\textwidth}{Xlll} \toprule
\emph{Object}          & \emph{Address Bits} & \emph{Level} & \emph{Methods} \\ \midrule
\texttt{PageDirectory} & 20---31             & 0            & \autoref{group__aarch32__seL4__ARM__PageDirectory} \\
\texttt{PageTable}     & 12---19             & 1            & \autoref{group__arm__seL4__ARM__PageTable} \\
\bottomrule
\end{tabularx}

\begin{tabularx}{\textwidth}{Xlll} \toprule
  \emph{Object}          & \emph{Address Bits} & \emph{Level} & \emph{Methods} \\ \midrule
  \texttt{PageDirectory (hyp)} & 21---31     & 0            & \autoref{group__aarch32__seL4__ARM__PageDirectory} \\
  \texttt{PageTable (hyp)}     & 12---20     & 1            & \autoref{group__arm__seL4__ARM__PageTable} \\
  \bottomrule
\end{tabularx}

\subsubsection{AArch64}

Depending on configuration, Arm AArch64 processors have page-table structures with 3 or 4 levels.
The VSpace object is \texttt{seL4\_ARM\_VSpaceObject}, which is a distinct object type used for the
top level page table. All intermediate paging structures are indexed by 9 bits of the virtual
address and are \obj{PageTable} objects. Depending on configuration, the top-level object is
indexed by either 9 or 10 bits. The macro \texttt{seL4\_VSpaceIndexBits} makes this value available
under a generic name. The table below shows the four-level configuration.

\begin{tabularx}{\textwidth}{Xlll} \toprule
\emph{Object}                    & \emph{Address Bits} & \emph{Level} & \emph{Methods} \\ \midrule
    \texttt{seL4\_ARM\_VSpaceObject}
                                 & 39---47             & 0            & \autoref{group__aarch64__seL4__ARM__VSpace} \\
    \texttt{PageTable}           & 30---38             & 1            & \autoref{group__arm__seL4__ARM__PageTable} \\
    \texttt{PageTable}           & 21---29             & 2            & \autoref{group__arm__seL4__ARM__PageTable} \\
    \texttt{PageTable}           & 12---20             & 3            & \autoref{group__arm__seL4__ARM__PageTable} \\
\bottomrule
\end{tabularx}

\subsection{RISC-V}

RISC-V provides the same paging structure for all levels, \obj{PageTable}. This means the VSpace
object is here also implemented by the \obj{PageTable} object.

\subsubsection{RISC-V 32-bit}

32-bit RISC-V \obj{PageTables} are indexed by 10 bits of virtual address.

\begin{tabularx}{\textwidth}{Xlll} \toprule
\emph{Object}          & \emph{Address Bits} & \emph{Level} & \emph{Methods} \\ \midrule
\texttt{PageTable}     & 22---31             & 0            & \autoref{group__riscv__seL4__RISCV__PageTable} \\
\texttt{PageTable}     & 12---21             & 1            & \autoref{group__riscv__seL4__RISCV__PageTable} \\
\bottomrule
\end{tabularx}

\subsubsection{RISC-V 64-bit}

64-bit RISC-V follows the SV39 model, where \obj{PageTables} are indexed by 9 bits of virtual address.
Although RISC-V allows
for multiple different numbers of paging levels, currently seL4 only supports exactly three levels
of paging structures.

\begin{tabularx}{\textwidth}{Xlll} \toprule
\emph{Object}          & \emph{Address Bits} & \emph{Level} & \emph{Methods} \\ \midrule
\texttt{PageTable}     & 30---38             & 0            & \autoref{group__riscv__seL4__RISCV__PageTable} \\
\texttt{PageTable}     & 21---29             & 1            & \autoref{group__riscv__seL4__RISCV__PageTable} \\
\texttt{PageTable}     & 12---20             & 2            & \autoref{group__riscv__seL4__RISCV__PageTable} \\
\bottomrule
\end{tabularx}

\subsection{Page}

\obj{Frame} objects, used via \obj{Page} capabilities, correspond to frames of physical memory that
are used to implement virtual memory pages in a virtual address space.

The virtual address for a \obj{Page} mapping must be aligned to the size of the \obj{Page} and must
be mapped into a suitable paging structure object, which itself must already be mapped in.

To map a page readable, the corresponding \obj{Page} capability must have read permissions. To map
the page writeable, the capability must have write permissions. The requested mapping permissions
are specified with an argument of type \texttt{seL4\_CapRights} given to the mapping invocation. If
the capability does not have sufficient permissions to authorise the given mapping, the mapping
permissions are silently downgraded. Specific mapping permissions are dependent on the architecture
and are documented in the \autoref{sec:api_reference} for each function.

At minimum, each architecture defines \texttt{Map}, \texttt{Unmap} and
\texttt{GetAddress} methods for pages.
Invocations for page capabilities for each architecture can be found in the \autoref{sec:api_reference}, and
are indexed per architecture in the table below.

\begin{tabularx}{\textwidth}{Xl} \toprule
\emph{Architectures} & \emph{Methods} \\ \midrule
IA32, X64            & \autoref{group__x86__seL4__X86__Page} \\
AArch32, AArch64     & \autoref{group__arm__seL4__ARM__Page} \\
    RISC-V           & \autoref{group__riscv__seL4__RISCV__Page} \\
\bottomrule
\end{tabularx}

Each architecture also defines a range of page sizes. In the next section we show the available page
sizes, as well as the \emph{mapping level}, which refers to
the level of the paging structure at which this page must be mapped.

\subsubsection{AArch32 page sizes}

\begin{tabularx}{\textwidth}{Xll} \toprule
    \emph{Constant}             & \emph{Size} & \emph{Mapping level} \\ \midrule
    \texttt{seL4\_PageBits}      & 4\,KiB      & 1                   \\
    \texttt{seL4\_LargePageBits} & 64\,KiB     & 1                   \\
    \texttt{seL4\_SectionBits}   & 1\,MiB      & 0                   \\
    \texttt{seL4\_SuperSectionBits} & 16\,MiB  & 0                   \\
    \bottomrule
\end{tabularx}

Mappings for sections and super sections consume 16 slots in the page table and page directory
respectively.

\subsubsection{AArch64 page sizes}

\begin{tabularx}{\textwidth}{Xll} \toprule
    \emph{Constant}             & \emph{Size} & \emph{Mapping level} \\ \midrule
    \texttt{seL4\_PageBits}      & 4\,KiB      & 3                   \\
    \texttt{seL4\_LargePageBits} & 2\,MiB     & 2                    \\
    \texttt{seL4\_HugePageBits}  & 1\,GiB     & 1                    \\
    \bottomrule
\end{tabularx}

\subsubsection{IA-32 page sizes}

\begin{tabularx}{\textwidth}{Xll} \toprule
    \emph{Constant}             & \emph{Size} & \emph{Mapping level} \\ \midrule
    \texttt{seL4\_PageBits}      & 4\,KiB      & 1                   \\
    \texttt{seL4\_LargePageBits} & 4\,MiB      & 0                   \\
    \bottomrule
\end{tabularx}

\subsubsection{X64 page sizes}

\begin{tabularx}{\textwidth}{Xll} \toprule
    \emph{Constant}             & \emph{Size} & \emph{Mapping level} \\ \midrule
    \texttt{seL4\_PageBits}      & 4\,KiB      & 3                   \\
    \texttt{seL4\_LargePageBits} & 2\,MiB      & 2                   \\
    \texttt{seL4\_HugePageBits}  & 1\,GiB      & 1                   \\
    \bottomrule
\end{tabularx}

\subsubsection{RISC-V 32-bit page sizes}

\begin{tabularx}{\textwidth}{Xll} \toprule
    \emph{Constant}             & \emph{Size} & \emph{Mapping level} \\ \midrule
    \texttt{seL4\_PageBits}      & 4\,KiB      & 1                   \\
    \texttt{seL4\_LargePageBits} & 4\,MiB      & 0                   \\
    \bottomrule
\end{tabularx}

\subsubsection{RISC-V 64-bit page sizes}

\begin{tabularx}{\textwidth}{Xll} \toprule
    \emph{Constant}             & \emph{Size} & \emph{Mapping level} \\ \midrule
    \texttt{seL4\_PageBits}      & 4\,KiB      & 2                   \\
    \texttt{seL4\_LargePageBits} & 2\,MiB      & 1                   \\
    \texttt{seL4\_HugePageBits}  & 1\,GiB      & 0                   \\
    \bottomrule
\end{tabularx}

\subsection{ASID Control}

The kernel supports a fixed maximum number of address space identifiers (ASIDs), which is
architecture dependent. In order to manage this limited resource, seL4 provides an \obj{ASID
Control} capability. The \obj{ASID Control} capability can be used together with an \obj{Untyped}
capability to create \obj{ASID pool} objects and capabilities, which authorise the use of a subset
of available address space identifiers. \obj{ASID Control} has a single \texttt{MakePool} method for
each architecture, listed in the table below.

\begin{tabularx}{\textwidth}{Xl} \toprule
\emph{Architectures} & \emph{Methods} \\ \midrule
IA32, X64            & \autoref{group__x86__seL4__X86__ASIDControl} \\
AArch32, AArch64     & \autoref{group__arm__seL4__ARM__ASIDControl} \\
RISC-V               & \autoref{group__riscv__seL4__RISCV__ASIDControl} \\
\bottomrule
\end{tabularx}

\subsection{ASID Pool}

An \obj{ASID Pool} confers the right to use a subset of the globally available address space
identifiers. The size of this subset is architecture dependent. For a VSpace object to be usable by
a thread, it must be assigned to an ASID via an \obj{ASID Pool} capability. Each ASID can be
assigned to at most one VSpace. The \obj{ASID Pool} capability has a single invocation,
\texttt{Assign}, for each architecture.

\begin{tabularx}{\textwidth}{Xl} \toprule
\emph{Architectures} & \emph{Methods} \\ \midrule
IA32, X64            & \autoref{group__x86__seL4__X86__ASIDPool} \\
AArch32, AArch64     & \autoref{group__arm__seL4__ARM__ASIDPool} \\
RISC-V               & \autoref{group__riscv__seL4__RISCV__ASIDPool} \\
\bottomrule
\end{tabularx}

\section{Mapping Attributes}
A parameter of type \texttt{seL4\_ARM\_VMAttributes}, \texttt{seL4\_x86\_VMAttributes}
, or \texttt{seL4\_RISCV\_VMAttributes} is used to specify the cache behaviour of the
page being mapped. Possible values for Arm that can be bitwise OR'd together are
shown in \autoref{tbl:vmattr_arm} \ifxeightsix. An enumeration of valid values
for IA-32 and x64 are shown in \autoref{tbl:vmattr_x86}\fi. Possible values for RISC-V that
can be bitwise OR'd together are shown in \autoref{tbl:vmattr_riscv}. Mapping attributes
can be updated on existing mappings using the Map invocation with the same virtual address.

\begin{table}[htb]
  \begin{center}
    \begin{tabularx}{\textwidth}{p{0.33\textwidth}X}
      \toprule
      Attribute & Meaning \\
      \midrule
      \texttt{seL4\_ARM\_PageCacheable} & Enable data in this mapping
      to be cached \\
      \texttt{seL4\_ARM\_ParityEnabled} & Enable parity checking for
      this mapping (ignored on AArch64) \\
      \texttt{seL4\_ARM\_ExecuteNever} & Map this memory as non-executable \\
      \bottomrule
    \end{tabularx}
    \caption{\label{tbl:vmattr_arm} Virtual memory attributes for Arm page
      table entries.}
  \end{center}
\end{table}

\begin{table}[htb]
  \begin{center}
    \begin{tabularx}{\textwidth}{p{0.33\textwidth}X}
      \toprule
      Attribute & Meaning \\
      \midrule
      \texttt{seL4\_x86\_WriteBack} & Read and writes are cached \\
      \texttt{seL4\_x86\_CacheDisabled} & Prevent data in this mapping
      from being cached \\
      \texttt{seL4\_x86\_WriteThrough} & Enable write through cacheing for this mapping \\
      \texttt{seL4\_x86\_WriteCombining} & Enable write combining for this mapping \\
      \bottomrule
    \end{tabularx}
    \caption{\label{tbl:vmattr_x86} Virtual memory attributes for x86 page
      table entries.}
  \end{center}
\end{table}

\begin{table}[htb]
  \begin{center}
    \begin{tabularx}{\textwidth}{p{0.33\textwidth}X}
      \toprule
      Attribute & Meaning \\
      \midrule
      \texttt{seL4\_RISCV\_ExecuteNever} & Map this memory as non-executable \\
      \bottomrule
    \end{tabularx}
    \caption{\label{tbl:vmattr_riscv} Virtual memory attributes for RISC-V page
      table entries.}
  \end{center}
\end{table}

\section{Sharing Memory}

The seL4 kernel does not allow intermediate paging structures (e.g.\ \obj{PageTable} objects) to be
shared, but it does allow pages to be shared between VSpaces, and VSpaces to be shared by threads.

To share a page, the capability to the \obj{Page} must first be duplicated using the
\apifunc{seL4\_CNode\_Copy}{cnode_copy} method and the copy must be used in the Map invocation
(e.g.\ \apifunc{seL4\_ARM\_Page\_Map}{arm_page_map} \ifxeightsix or
\apifunc{seL4\_x86\_Page\_Map}{x86_page_map}\fi) that maps the page into the second address space.
Attempting to map the same capability twice in different page tables or address spaces will result
in an error.


\section{Page Faults}

Page faults are reported to the exception handler of the executed thread.
See \autoref{sec:vm-fault}.
